\documentclass{article}

\usepackage{amsmath}
\usepackage{amssymb}

\newcommand{\bx}{{\bf x}}
\newcommand{\bbeta}{\boldsymbol\beta}

\begin{document}

Our predictor function for the number of riders at a given hour
(described by input parameters $\bx$) is
\[
c(\bx, \bbeta)
\]
where $\bbeta$ is our set of model parameters.

Our first model assumption separates commuting riders from
recreational riders; obviously, the total count is the
sum of their distinct counts:
\[
c(\bx, \bbeta) = c_c(\bx, \bbeta) + c_r(\bx, \bbeta)
\]
where $c_c$ and $c_r$ are commuting and recreating counts,
respectively.

We will assume that, given perfect weather and unchanging program
popularity, the counts will vary hour-to-hour on a daily cycle.
\[
c(\bx, \bbeta) = \alpha_c(\bx, \bbeta) \hat{c}_c(h) +
\alpha_r(\bx, \bbeta) \hat{c}_r(h)
\]
where $h$ is the hour of day (1-24), obtained from $\bx$.

\[
\hat{c}(h) =
\frac{
  \beta_{\hat{c},h}
}
{
  \sum_{i=1}^{24} \beta_{\hat{c},i}
}
\]

The gradient is
\begin{align}
\partial_{\beta_{\hat{c},j}} \hat{c}(h) &=
\frac{
  \delta_{h,j}
}{
  \sum_{i=1}^{24} \beta_{\hat{c},i}
} -
\frac{
  \beta_{\hat{c},h}
}{
  (\sum_{i=1}^{24} \beta_{\hat{c},i})^2
} \\
&=
\frac{
  \delta_{h,j} \sum_{i=1}^{24} \beta_{\hat{c},i} -
  \beta_{\hat{c},h}
}{
  (\sum_{i=1}^{24} \beta_{\hat{c},i})^2
}
\end{align}



%\begin{align}
%\partial_{\beta_{\hat{c},j}} \hat{c}(h) &=
%\sum_{i=1}^{23} \delta_{i,j} \delta_{h,i}
%- \sum_{i=1}^{23} \delta_{i,j} \delta_{h,24} \\
%&= \delta_{h,j} - \delta_{h,24}
%\end{align}
and since $j \neq 24$

\subsection*{Weather effect}

Introduce the weather effect function, $f(W, \bbeta) \in [0, 1]$, where the
weather score, $W$, given in $\bx$, can take on only 4 values
(integers 1-4).  The weather score introduces a multiplicative effect under
the assumption that bad weather will reduce ridership by a certain
percentage of the fair-weather number. For simplicity (at first),
the weather score will affect both commuters and recreators equally.
\[
c(\bx, \bbeta) = f(W, \bbeta_f) \left( \alpha^{\prime}_c(\bx, \bbeta) \hat{c}_c(h) +
\alpha^{\prime}_r(\bx, \bbeta) \hat{c}_r(h) \right)
\]

where $\beta_n \in [0, 1]$.
Removing amplitude from the weather effect gives
\[
f(W, \bbeta_f) = \delta_{W,1} + \beta_{f,W-1}
\]

$W=1$ represents ``ideal''
weather (no rain, storms, etc).

The gradient is
\[
\partial_{\beta_{f,i}} f = \delta_{W-1,i}
\]


A smarter approach might smooth (average) $W$ 
over several hours, under the assumption that someone caught in a
flash storm on their bike may keep riding or take cover until the
storm passes.

If $W$ is smoothed to $\bar{W}$, a continuous function of $\bar{W}$
must be approximated, for example with polynomials,
\[
f(W, \bbeta) \approx \sum_{n=1}^N \beta_n \bar{W}^{n-1}
\]

In either case, we label the number of parameters needed by the
approximated weather effect by $N$ ($N=4$ in the simple delta-function
approximation).

\subsection*{Day type}

We also introduce a {\it day-type} effect, $g(D, \bbeta_g)$, which
considers differences in ridership patterns for working days,
holidays, weekends, etc. We assign an integer value $D \in [0, M]$ for each of these
day types.
\[
g(D, \bbeta_g) = \delta_{D,0} + (1 - \delta_{D,0}) \beta_{g,D}
\]
where $\beta_{g,D} \in [0, \infty)$. Arbitrarily choose $D = 0$ to be the
working day type.

The gradient is
\[
\partial_{\beta_{g,i}} g(D, \bbeta_g) = \delta_{D,i}
\]

Day type will affect recreational and commuter ridership differently,
therefore, we need two day-type effect functions, $g_c$ and $g_r$.

\subsection*{Temperature}

Each input vector $\bx_i$ contains a continuous temperature variable
$T_i$.
Introduce a temperature effect $f_T(T, \bbeta_T)$
that multiplies our count. At an
``ideal'' temperature $\bar{T}$, we have the constraint
\[
f_T(\bar{T}, \bbeta_T) = 1
\]
additionally, we enforce the boundary constraints
\[
f_T(-\infty, \bbeta_T) = f_T(\infty, \bbeta_T) = 0
\]

We implement an asymmetric Gaussian approximation
\[
f_T(T, \bbeta_T) =
\begin{cases}
  e^{-(T - \bar{T})^2 / 2 \Delta T_{\rm low}^2} & T < \bar{T} \\
  e^{-(T - \bar{T})^2 / 2 \Delta T_{\rm hi}^2} & T \geq \bar{T}
\end{cases}
\]
where $\bbeta_T = [\bar{T}, \Delta T_{\rm low}, \Delta T_{\rm hi}]$,
$\Delta T_{\rm low} \geq 0$, and
$\Delta T_{\rm hi} \geq 0$.

The gradient is
\[
\partial_{\bar{T}} f_T =
\begin{cases}
  (T - \bar{T})
    e^{-(T - \bar{T})^2 / 2 \Delta T_{\rm low}^2} / \Delta T_{\rm low}^2
    & T < \bar{T} \\
  (T - \bar{T})
    e^{-(T - \bar{T})^2 / 2 \Delta T_{\rm hi}^2} / \Delta T_{\rm hi}^2
    & T \geq \bar{T}
\end{cases}
\]
and
\[
\partial_{T_{\rm low}} f_T =
\begin{cases}
  (T - \bar{T})^2
    e^{-(T - \bar{T})^2 / 2 \Delta T_{\rm low}^2} /
    \Delta T_{\rm low}^3
    & T < \bar{T} \\
  0 & T \geq \bar{T}
\end{cases}
\]
and
\[
\partial_{T_{\rm hi}} f_T =
\begin{cases}
  0 & T < \bar{T} \\
  (T - \bar{T})^2
    e^{-(T - \bar{T})^2 / 2 \Delta T_{\rm hi}^2} / \Delta T_{\rm hi}^3
    & T \geq \bar{T}
\end{cases}
\]

%\[
%f_T(T, \bbeta_T) =
%\begin{cases}
%  0 & T < T_{\rm low} \\
%  \frac{1}{2} (1 - \cos(\pi (T - T_{\rm low}) /
%  2 \Delta T_{\rm low})) & T_{\rm low} < T < \bar{T} \\
%  \frac{1}{2} (1 - \cos(\pi (T - \bar{T}) /
%  2 \Delta T_{\rm hi})) & \bar{T} < T < T_{\rm hi} \\
%  0 & T > T_{\rm hi}
%\end{cases}
%\]
%where $\bbeta_T = [\bar{T}, \Delta T_{\rm low}, \Delta T_{\rm hi}]$,
%$\Delta T_{\rm low} \geq 0$,
%$\Delta T_{\rm hi} \geq 0$,
%$T_{\rm low} = \bar{T} - \Delta T_{\rm low}$,
%and $T_{\rm hi} = \bar{T} + \Delta T_{\rm hi}$.

%The gradient is
%\[
%\partial_{\bar{T}} f_T =
%\begin{cases}
%  0 & T < T_{\rm low} \\
%  \frac{1}{2} \sin(\pi (T - T_{\rm low}) / 2 \Delta T_{\rm low})) & T_{\rm low} < T < \bar{T} \\
%  \frac{1}{2} (1 - \cos(\pi (T - \bar{T}) /
%  2 \Delta T_{\rm hi})) & \bar{T} < T < T_{\rm hi} \\
%  0 & T > T_{\rm hi}
%\end{cases}
%\]

\subsection*{Popularity}

Finally, a {\it popularity} amplitude $p(t, \bbeta)$ determines the amplitude of our
count predictor function, where $t$ is the date (day).
\[
p(t, \bbeta_p) = \beta_{p,t}
\]
And the gradient is
\[
\partial_{\beta_{p,i}} p(t, \bbeta_p) = \delta_{t,i}
\]


\subsection*{Generic model}

Let us assume first a popularity trend. This is the number of people
that know about the service at any given time (let's further assume
a daily granularity for the popularity trend; that is, the popularity
is constant over the course of a day). The popularity trend must
increase or stay the same from day to day. This trend is not given by
the training data; only the ridership at each hour of each day is
given. So our model starts out depending on only the date:
\[
c(\bx) = c(d)
\]
where $d$ is the date and
\[
c(d + \Delta d) \geq c(d)
\]
where $\Delta d > 0$.

Next, assume that the hour of day is a strong indicator of ridership
for a given popularity.

\[
c \approx f(d, h, T, w, m, q)
\]

First, assume that the baseline ridership only increases from day to
day; i.e.
\[
\partial_d f \geq 0
\]
Also assume that the above is ``small'', that is, baseline ridership
for a given time of day does not change much day to day (assuming all
other inputs are constant).

The day type, $q$, is not a continuous variable, therefore, for
maximum generality, we treat it
as an index.
\[
c \approx \sum_q f_q(d, h, T, w, m)
\]
There is likely a relationship between certain day types (e.g. a
Tuesday is probably similar to a Wednesday in the same week, all else
constant).



\subsection*{Full model}

The model with the fewest parameters gives identical weather and
day-type effects to both recreational and commuting riders
\[
c(\bx, \bbeta) = p(t, \bbeta_p) f(W, \bbeta_f)
\big( g_c(D, \bbeta_{g_c}) \hat{c}_c(h, \bbeta_{c_c}) + g_r(D,
\bbeta_{g_r}) \hat{c}_r(h, \bbeta_{c_r}) \big)
\]

The above model results in $P$ parameters,
\[
P = N + 2 M
\]


The important inputs are
\begin{enumerate}
  \item Rider type (2 values)
  \item Hour of day (24 values)
  \item Weather (3 or $N$ values)
  \item Day type ($M$ values)
\end{enumerate}
so the total size of the input space (in the simple weather case) is
\[
2 \times 24 \times N \times M = 48 N M
\]

In the simple case $N=3$ (3 non-normal weather values), $M=2$
(weekend, holiday, working day), the input space has dimension
288.

\section*{Training}

We will brute force the minimization of
\[
e = \sqrt{\frac{1}{n} \sum_i^n (\log (1 + c_i) - \log(1 + c(\bx_i,
\bbeta)))^2}
\]

The analytic gradient of the cost function may be useful.
\[
\nabla_{\bbeta} e = - \frac{1}{e n} \sum \frac{e_i}{1 + c(\bx_i,
\bbeta)} \nabla_{\bbeta} c(\bx_i, \bbeta)
\]
We can split up our gradient by effect.
\[
\nabla_{\bbeta} =
\nabla_{\bbeta_{\hat{c}_c}} +
\nabla_{\bbeta_{\hat{c}_r}} +
\nabla_{\bbeta_{g_c}} +
\nabla_{\bbeta_{g_r}} +
\nabla_{\bbeta_{p}} +
\nabla_{\bbeta_{f}}
\]

\begin{align}
\nabla_{\bbeta_{\hat{c}_r}} c &=
p(t, \bbeta_p) f(W, \bbeta_f) g_r(D, \bbeta_{g_r}) \bbeta_{\hat{c}_r}
\\
\nabla_{\bbeta_{\hat{c}_c}} c &=
p f g_c \bbeta_{\hat{c}_c}
\\
\nabla_{\bbeta_{g_c}} c &=
p f g_c \bbeta_{g_c}
\end{align}

\section*{Predictions}

We would like to predict $c(\bx_n)$ given $c(\bx_1)$ through
$c(\bx_{n-1})$. We assume that $p(t)$ does not change much over the
course of several days.

Because the $p(t)$ and $c_{(c,r)}(h)$ terms in our model do not
contain parameters,


We assume the general form:
\[
c(\bx, \bbeta) = \alpha_1(W, T, D) \sum_n f_n(t) + \alpha_2(W, T, D) f_2(t)
\]
where $t$ is in hours, $W$ is the weather score (1-4), $T$ is the
temperature, $D$ is the type of day (holiday, regular working day, or
weekend day), $f_1$ is the curve for commuting riders, and $f_2$
is the curve for recreational riders. We assume (given the obvious
daily shape of the data) that $f_n(t + 24) \approx f_n(t)$.

Fluctuations in the ``normal'' shapes $f_n$ are due to environmental
factors and a little bit of randomness (like getting sick and staying
home from work).

We would like to predict $c(t)$ given $c(t^{\prime} < t)$.

The training data contains known values for $c(t)$ for ranges of $t$
at the first 2/3 of every month.


We need a model for the shape function coefficients. Parameterize with
$\beta$.
\[
\alpha_n(\bx, \bbeta) \approx \alpha_n(\bx, \bbeta_0) + \bbeta \cdot \nabla
\alpha_n(\bx, \bbeta) |_{\bbeta = \bbeta_0}
\]

\[
\alpha_n(\bx, \bbeta) = \prod_i (\beta_{i,1} + \beta_{i,2} \Xi(\bx))
\]

Consider the difference,
\[
c(t) - c(t - 24) \approx
(\alpha_1(\bx(t)) - \alpha_1(\bx(t - 24))) f_1(t - 24) +
(\alpha_2(\bx(t)) - \alpha_2(\bx(t - 24))) f_2(t - 24)
\]

\section*{Model}

A data sample comes at the start of every hour, $i$. A known rider
count is $c_i$ while a predicted one is $\tilde{c}_i$.

Each unknown rider count is a function of parameters $\beta_i$, given the
training dataset.

\[
\tilde{c}_i = \tilde{c}(\bx_i, \bbeta)
\]

Assume the form

\[
\tilde{c}_i = \frac{\sum_{j=1}^{i-1} w_{i,j} c_j}{\sum_{j=1}^{i-1}
w_{i,j}}
\]


Let $h(i)$ be the hour-of-day for sample $i$, $W(i)$ return 1 if
sample $i$ was taken during a workday and 0 otherwise. We also assume that

\[
w_{i,j} \propto \delta_{h(i),h(j)} (\alpha \delta_{W(i),W(j)} + \gamma (1 - \delta_{W(i),W(j)}))
\]

For known counts, we measure the accuracy of a prediction for that
count (given preceding historical data) as

\[
e_i = \log(1 + c_i) - \log(1 + \tilde{c}_i)
\]

and the total error is

\[
e = \sqrt{\frac{1}{N} \sum_{i=1}^N e_i^2}
\]


\section*{Correlation study}

Measure the correlations in the data by

\[
C = \sum_i c_i - c_{i+j}
\]

\end{document}
